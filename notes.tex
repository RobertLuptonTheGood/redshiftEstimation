\documentclass[12pt]{article}
%\usepackage{graphicx}
\usepackage[top=1cm, bottom=2cm, left=2cm, right=2cm]{geometry} 
\usepackage{xspace}
\usepackage{perpage}
%
\renewcommand*{\thefootnote}{\fnsymbol{footnote}}
\MakePerPage{footnote}
%
\begin{document}
\title{What is a Photometric Redshift?}
\author{Brice M\'enard and Robert Lupton}
%\date{1996-03-06}
\maketitle

\newcommand{\eg}{\textit{e.g.}\xspace}
\newcommand\Fi{\{F_i\}}
\newcommand\FdFi{\{F_i, dF_i\}}
\newcommand{\ie}{\textit{i.e.}\xspace}
\newcommand{\photoz}{z_p\xspace}
\newcommand{\clusterz}{z_c\xspace}

\section{Introduction}

In a formal sense photometric and spectroscopic measurements are the same; in both cases we make a number of
measurements of an object's flux $\Fi$, either a relatively small number (photometry) or a large or very
large number (spectroscopy).

Measuring an object's redshift from spectroscopic data is typically achieved by looking for reasonably-well
localised astrophysical features in $\Fi$ (\eg emission lines of hydrogen or oxygen; the calcium
triplet; the 4000\AA{} break) and directly deducing the redshift along with other characteristics of the
object such as its star formation rate or metallicity.  The continuum may or may not be neglected in this
analysis.

Broadband photometric data\footnote{Narrow-band datasets such as COMBO-17 are an intermediate case which we
  need not explicitly discuss here.} can be analysed in a similar way, looking for continuum features such as
the Balmer or Lyman break.  Unfortunately these features can seldom be unambigously identified from four or
five bands (and thus three or four colours), so rather than local measures, such as an emission line, the
entire spectrum is considered simultaneously and including the continuum in the analysis is essential.

In both cases we are defining a mapping from $\Fi$ to a redshift. For spectroscopic data
at high signal-to-noise and resolution the vector $\Fi$ may define a small-enough probability cloud in a
high-enough dimensional space that all objects in our sample are well separated.  In this case, it makes sense
to talk about the object's redshift.\footnote{Or redshifts; even in this ideal universe there can more than one
  possible redshift, the obvious example being a spectrum with a single unresolved emission line.}

When processing photometric data, or for spectroscopy of faint objects at relatively low effective dispersion
it may not be possible to unambiguously assign a redshift (is this feature [OII], [OIII] or Ly$_\alpha$?  Is
that broad break due to the Balmer or Lyman series?) and all we can do is to determine a probability
distribution $p(z)$.\footnote{Or, more precisely, a joint probability of its redshift and other properties.}

Once we also consider the errors in the measured fluxes we must consider all the redshifts consistent with our
measured $\FdFi$ resulting in a possibly-multimodal $p(z)$ whose form is partially intrinsic and partially due
to our measurement errors.  In reality this $p(z)$ does not apply to a given object, but merely to a class of
objects with statistically indistinguishable $\Fi$.

This ability to extract redshift information from photometric data allows us to extract
important astrophysical information from extragalactic astronomical datasets; in the context of large sky
surveys, the information content in $p(z | \Fi)$ has even become a metric to best define some of the basic photometric parameters of celestial sources.

\section{Measuring $p(z | \Fi)$}

How should we measure the mapping from flux to redshift?  There are a number of approaches:

\begin{itemize}
\item
  For each object in the sample we may attempt to fit a theoretical or observed spectrum to
  the available fluxes.

\item
  Given a training set of galaxies with known (presumably spectroscopic) redshifts we may use machine learning
  techniques to assign a redshift to each object in the sample.

\item
  As argued above, we are not really interested in the redshift of an individual object, but with
  those a set of indistinguishable objects.  We may use the method of `clustering redshifts' to
  estimate $p(z)$ directly for any cell in colour space.
\end{itemize}

\section{Properties of the Mapping between Colour and Redshift}

As we have emphasized, the quantity of interest is the mapping between the photometric and redshift spaces,
not the redshift estimate of a given object. This mapping has a number of properties:

\begin{itemize}
\item uniqueness: one to one or one to many mapping.
  This needs to be considered in both directions.

\item compactness: $dV/d\,\mbox{\it color} \rightarrow dV/dz$ where $V$ is the volume spanned by the sources in the
  global photometric or redshift space.

\item dimensionality of the photometric space. How large is it given a survey? While classical photometric redshifts focus primarily on colors, clustering redshifts can make use of all the dimensions of the photometric space.

\item inverse mapping: how does given a redshift range map onto the photometric space?

\item effects due to stellar contamination in the `galaxy' sample: the cross-correlation coefficient between the spatial position of photometric sources and a set of spectroscopic objects can be used to infer the proportions of stars and galaxies in the photometric sample. 

One way to proceed with this is to consider two sets of spatial cross-correlations with two spectroscopic
samples having different clustering bias.

\end{itemize}

Comments:

Having one characterization of such a mapping implies that:
\begin{itemize}
\item there is no need to store photo-z information per galaxy. This is relevant for the LSST database.
\item The structure of the horizontal stripes in a $d^2N/(d\clusterz d\photoz)$ diagram contains information on the stellar contamination as a function of $\photoz$ or any other parameter of consideration.

\item I will claim that one of the ultimate quests in observational extragalactic astronomy is to characterize and understand the mapping between the photometric space and redshift space. This is a fundamental `object' one might want to use to constrain any models of galaxy formation and cosmology from observations.
\end{itemize}

\end{document}
